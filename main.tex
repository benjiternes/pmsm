% ----------------------------------------------------------------------------------------------------------
% Praeambel
% ----------------------------------------------------------------------------------------------------------
\documentclass[fontsize=12pt,%
paper=a4,%
DIV=classic,%
BCOR=8mm,%
twoside=false,%
headings=openany,%
parskip=half,%
pagesize=auto,%
numbers=noenddot,%
headsepline=true,%
toc=listof,%
toc=bibliography%
]{scrreprt}

% PDF-Kompression
\pdfminorversion=5
\pdfobjcompresslevel=1

% Schriften
\usepackage{mathpazo,tgpagella}
%\usepackage{libertine}
%\usepackage{fourier}
\usepackage{lmodern}

% Allgemeines
\usepackage{amsmath,amssymb} % Mathesachen
\usepackage[T1]{fontenc} % Ligaturen, richtige Umlaute im PDF
\usepackage[utf8]{inputenc}% UTF8-Kodierung für Umlaute usw

\usepackage{siunitx}
\usepackage{scrlayer-scrpage} 
\pagestyle{scrheadings}


%\usepackage{setspace}
%\onehalfspacing 

% Schriften-Größen
\setkomafont{chapter}{\Huge\rmfamily}
\setkomafont{section}{\Large\rmfamily}
\setkomafont{subsection}{\large\rmfamily}
\setkomafont{subsubsection}{\large\rmfamily}
\setkomafont{chapterentry}{\large\rmfamily} % Überschrift der Ebene in Inhaltsverzeichnis
\setkomafont{descriptionlabel}{\bfseries\rmfamily} % für description Umgebungen
\setkomafont{captionlabel}{\small\bfseries}
\setkomafont{caption}{\small}

% Sprache: Deutsch
\usepackage[autostyle,babel,german=guillemets,style=german]{csquotes}
\usepackage[USenglish,ngerman]{babel} 
\selectlanguage{ngerman}
% PDF
\usepackage[ngerman,%
pdfauthor={B. Ternes},%
pdftitle={Modellbasierte Implementierung einer Vektorregelung für Synchronmaschinen},%
colorlinks=true,linkcolor=blue,citecolor=blue,filecolor=magenta,urlcolor=blue%
]{hyperref}

% BibLaTeX
\usepackage[backend=biber,%
 style=alphabetic,%
 autocite=inline,%
 sorting=anyt,%
 sortcites=true,%
 hyperref=auto,%
 maxnames=2,%
 minnames=1,%
]{biblatex}

\bibliography{bibo.bib}

\setlength{\bibitemsep}{1em}
\DefineBibliographyStrings{ngerman}{andothers = {u.\,a.},}

\usepackage[final]{microtype} % mikrotypographische Optimierungen
\usepackage{url}
\usepackage{pdflscape} % einzelne Seiten drehen können

% Tabellen
\usepackage{multirow} % Tabellen-Zellen über mehrere Zeilen
\usepackage{multicol} % mehre Spalten auf eine Seite
\usepackage{tabularx} % Für Tabellen mit vorgegeben Größen
\usepackage{longtable} % Tabellen über mehrere Seiten
\usepackage{array}

%  Bibliographie
%\usepackage{bibgerm} % Umlaute in BibTeX

% Tabellen
\usepackage{multirow} % Tabellen-Zellen über mehrere Zeilen
\usepackage{multicol} % mehre Spalten auf eine Seite
\usepackage{tabularx} % Für Tabellen mit vorgegeben Größen
\usepackage{array}
\usepackage{float}

% Bilder
\usepackage{graphicx}
\usepackage{color}
\graphicspath{{_Bilder/}}

\DeclareGraphicsExtensions{.pdf,.png,.jpg}
\usepackage{subfigure}
\newcommand{\subfigureautorefname}{\figurename}
\usepackage[all]{hypcap}

% Bildunterschrift
\setcapindent{0em}
\setcapwidth[c]{0.9\textwidth}
\setlength{\abovecaptionskip}{0.2cm}

% Quellcode
\usepackage{listings}
\definecolor{grau}{gray}{0.25}
\lstset{
	extendedchars=true,
	basicstyle=\tiny\ttfamily,
	%basicstyle=\footnotesize\ttfamily,
	tabsize=2,
	keywordstyle=\textbf,
	commentstyle=\color{grau},
	stringstyle=\textit,
	numbers=left,
	numberstyle=\tiny,
	% für schönen Zeilenumbruch
	breakautoindent  = true,
	breakindent      = 2em,
	breaklines       = true,
	postbreak        = ,
	prebreak         = \raisebox{-.8ex}[0ex][0ex]{\Righttorque},
}

% linksbündige Fußboten
\deffootnote{1.5em}{1em}{\makebox[1.5em][l]{\thefootnotemark}}

\typearea{14} % typearea berechnet einen sinnvollen Satzspiegel (das heißt die Seitenränder) siehe auch http://www.ctan.org/pkg/typearea. Diese Berechnung befindet sich am Schluss, damit die Einstellungen oben berücksichtigt werden
% für autoref von Gleichungen in itemize-Umgebungen
\makeatletter
\newcommand{\saved@equation}{}
\let\saved@equation\equation
\def\equation{\@hyper@itemfalse\saved@equation}
\makeatother 

\usepackage[framemethod=TikZ]{mdframed}
\usepackage{xcolor}

\definecolor{light-gray}{gray}{0.85}

\newmdenv[%
    backgroundcolor=light-gray,
    linecolor=light-gray,
    outerlinewidth=2pt,
    roundcorner=2mm,
    skipabove=\baselineskip,
    skipbelow=\baselineskip,
]{shaded}
\newcommand{\mat}[1]{
      {\textbf{#1}}
}

\newcommand{\todo}[1]{
      {\colorbox{red}{ TODO: #1 }}
}

\newcommand{\todotext}[1]{
      {\color{red} TODO: #1} \normalfont
}

\newcommand{\info}[1]{
      {\colorbox{blue}{ (INFO: #1)}}
}

% Hinweis auf Programme in Datei
\newcommand{\datei}[1]{
      {\ttfamily{#1}}
}
\newcommand{\code}[1]{
      {\ttfamily{#1}}
}
% bild mit defnierter Breite einfügen
\newcommand{\bild}[4]{
  \begin{figure}[!hbt]
    \centering
      \vspace{1ex}
      \includegraphics[width=#2]{images/#1}
      \caption[#4]{\label{img.#1} #3}
    \vspace{1ex}
  \end{figure}
}
% bild mit eigener Breite
\newcommand{\bilda}[3]{
  \begin{figure}[!hbt]
    \centering
      \vspace{1ex}
      \includegraphics{images/#1}
      \caption[#3]{\label{img.#1} #2}
      \vspace{1ex}
  \end{figure}
}


% Bild todo
\newcommand{\bildt}[2]{
  \begin{figure}[!hbt]
    \begin{center}
      \vspace{2ex}
	      \includegraphics[width=6cm]{images/todobild}
      %\caption{\label{#1} \color{red}{ TODO: #2}}
      \caption{\label{#1} \todotext{#2}}
      %{\caption{\label{#1} {\todo{#2}}}}
      \vspace{2ex}
    \end{center}
  \end{figure}
}

\usepackage{xspace} 
\newcommand{\AdV}{Anm.\ d.\ Verf.\@\xspace}
\newcommand{\bzw}{bzw.\@\xspace}
\newcommand{\etc}{etc.\@\xspace}
\newcommand{\bspw}{bspw.\@\xspace}
\newcommand{\bzgl}{bzgl.\@\xspace}
\newcommand{\ea}{et\,al.\@\xspace}
\newcommand{\etal}{et\,al.\@\xspace} 
\newcommand{\latex}{\LaTeX\@\xspace}
\newcommand{\insb}{insbes.\@\xspace}
\newcommand{\dH}{d.\,h.\@\xspace}
\newcommand{\ggf}{ggf.\@\xspace}
\newcommand{\ggfs}{ggfs.\@\xspace}
\newcommand{\idr}{i.\,d.\,R.\@\xspace}
\newcommand{\ivm}{i.\,V.\,m.\@\xspace}
\newcommand{\ieS}{i.\,e.\,S.\@\xspace}
\newcommand{\is}{i.\,S.\@\xspace}
\newcommand{\iwS}{i.\,w.\,S.\@\xspace}
\newcommand{\oae}{o.\,{\"a}.\@\xspace}
\newcommand{\og}{o.\,g.\@\xspace}
\newcommand{\ua}{u.\,a.\@\xspace}
\newcommand{\uae}{u.\,{\"a}.\@\xspace}
\newcommand{\usw}{usw.\@\xspace}
\newcommand{\uU}{u.\,U.\@\xspace}
\newcommand{\uvm}{u.\,v.\,m.\@\xspace}
\newcommand{\va}{v.\,a.\@\xspace}
\newcommand{\zB}{\mbox{z.\,B.}\@\xspace}
\newcommand{\zt}{z.\,T.\@\xspace}
\newcommand{\vglzb}{vgl.\,z.\,B.\@\xspace}
\newcommand{\vgl}{vgl.\@\xspace}
\newcommand{\sog}{sog.\@\xspace}
\newcommand{\so}{s.\,o.\@\xspace}
\newcommand{\msc}{M.\,Sc.\@\xspace}
\newcommand{\bsc}{B.\,Sc.\@\xspace} 
	
\usepackage{pdfpages}

\renewcommand{\i}[1]{\mathrm{#1}}
\newcommand{\x}[1]{\mathrm{#1}}

\newcommand{\unit}[1]{$\mathrm{#1}$}
\newcommand{\sh}{s.~h.~}
\newcommand{\productname}[1]{\textsc{#1}}

\newcommand{\longpage}{\enlargethispage{\baselineskip}}
\newcommand{\verylongpage}{\enlargethispage{2\baselineskip}}
\newcommand{\shortpage}{\enlargethispage{-\baselineskip}}
\newcommand{\veryshortpage}{\enlargethispage{-2\baselineskip}}

\begin{document}
\pagenumbering{Roman}
\pagestyle{empty}

% ----------------------------------------------------------------------------------------------------------
% Titelseite
% ----------------------------------------------------------------------------------------------------------
\clearscrheadings\clearscrplain

\begin{center}
	\begin{minipage}{7cm}
	\textsc{Fachbereich Elektrotechnik\\
	und Informatik}\\
	\\
	Institut für Systemtechnik
	\end{minipage}
	\hfill
	\begin{minipage}{7cm}
	\includegraphics[width=\textwidth]{bo}\\
	\end{minipage}
	
	\vspace*{4cm}
	
	\Huge
	\textbf{Projektarbeit}\\
	\vspace*{0.5cm}
	\large
	über das Thema\\
	\vspace*{1cm}
	\Huge
	\textbf{Modellbasierte Implementierung einer Vektorregelung für Synchronmaschinen}\\
	\vspace*{2cm}
	\vfill
	\normalsize
	\newcolumntype{x}[1]{>{\raggedleft\arraybackslash\hspace{0pt}}p{#1}}
	\begin{tabular}{x{6cm}p{7.5cm}}
		\rule{0mm}{5ex}\textbf{Autoren:} & Benjamin Ternes\newline benjamin.ternes@fernuni-hagen.de\newline Matrikelnummer: 014102076 \\
		\rule{0mm}{5ex}                & Jan Feldkamp\newline jan.feldkamp@hs-bochum.de \newline Matrikelnummer: 012215207\\ 
		\rule{0mm}{5ex}\textbf{Prüfer:} & Prof. Dr.-Ing. A. Bergmann \\ 
		\rule{0mm}{5ex}\textbf{Abgabedatum:} & \today \\ 
	\end{tabular} 
\end{center}
\clearpage

\pagestyle{scrheadings}
\ohead[]{\pagemark}
\ihead[]{\headmark}
\ifoot[]{}
\automark[chapter]{chapter}
\automark*[section]{}
\renewcommand*{\chapterpagestyle}{scrheadings}
% ----------------------------------------------------------------------------------------------------------
% Verzeichnisse
% ----------------------------------------------------------------------------------------------------------
\tableofcontents

%\listoftables

\chapter*{Symbolverzeichnis}\label{s.sym}
\addcontentsline{toc}{chapter}{Symbolverzeichnis}
\markboth{Symbolverzeichnis}{Symbolverzeichnis}
% ----------------------------------------------------------------------------------------------------------
% Symbolliste
% ----------------------------------------------------------------------------------------------------------
%\section*{Allgemeine Symbole}\label{s.sym.alg}
\begin{flushleft}
\begin{tabular}{lll}
\toprule
Symbol & Bedeutung	& Einheit\\
\midrule
$A$			&	elektrische Strombelag  	&  	\si{\ampere\per\meter} \\
$\hat{A}_\x{1}$			&	Amplitude des Strombelages  	&  	\si{\ampere\per\meter} \\ 
$B$			&	magnetische Flussdichte		&	\si{\volt\second\per\square\meter} \\
$\hat{B}_\x{1}$	 &	Amplitude der magnetischen Flussdichte 	&  	\si{\volt\second\per\square\meter} \\ 
$D$ & Verschiebungsstrom &  \si{\ampere} \\
$E$ & elektrische Feldstärke & \si{\volt\per\meter} \\
$H$			&	magnetische Feldstärke		&	\si{\ampere\per\meter}\\
$H_\x{S}$	&   magnetische Feldstärke in Nutöffnung	&	\si{\ampere} \\
$H_\x{t}$	&   Tangentiale Feldstärke	&	\si{\ampere} \\
$I_\i{1}$	&	Ständerstrom	&	\si{\ampere} \\
$I_\i{e}$	&	Erregerstrom	&	\si{\ampere} \\
$I_\i{\mu}$	&	Magnetisierungsstrom	&	\si{\ampere} \\
$J$			&	elektrische Stromdichte		&	\si{\ampere\per\square\meter} \\
$J_\x{m}$			&	Trägheitsmoment	&	\si{\kilogram\square\meter} \\
$L_\mathrm{e}$	& 	Erregerinduktivität & \si{\volt\second\per\ampere}\\
$L_\mathrm{d}, L_\mathrm{q}$	& 	Statorinduktivität im d,q-System & \si{\volt\second\per\ampere}\\
$M_\x{i}$			&	innere Drehmoment		&	\si{\newton\meter} \\
$M_\x{Last}$			&	Lastmoment		&	\si{\newton\meter} \\
$R_\mathrm{1}$		&	Statorwiderstand			&	\si{\ohm} \\
$R_\i{e}$	&	Erregerkreisverlustwiderstand &	\si{\ohm}\\
$T$ & Periodendauer & \si{\second} \\
$\underline{T}'$ & nicht quadratische Transformationsmatrix & \\
$\underline{T}$ & quadratische Transformationsmatrix & \\
$U_\i{1}$ & Ständerspannung & \si{\volt}\\
$U_\i{e}$ & Erregerspannung & \si{\volt}\\
$U_\i{p}$ & Polradspannung & \si{\volt}\\
$U_\i{h}$ & Hauptfeldspannung & \si{\volt}\\
$V$ &	magnetisches Vektorpotenzial&	\si{\volt\second\per\meter}\\
$V_\mathrm{\delta}$		&	magnetische Spannung		&	\si{\ampere} \\
$X$	& 	Reaktanz & \si{\ohm}\\
$X_\x{1}$	& 	Ständerreaktanz & \si{\ohm}\\
$X_\x{d}$	& 	synchron Reaktanz & \si{\ohm}\\
$X_\x{h}$	& 	Streureaktanz & \si{\ohm}\\

$\underline{a}$ & Drehoperator &  \si{\radian} \\
$b_\x{S}$	& Breite des Nutschlitzes & \si{\meter}\\
$d_\x{i}$	& Innendurchmesser & \si{\meter}\\
$f$ & Frequenz & \si{\per\second} \\
$p$		&	Polpaarzahl					&	\\
$u$ & zeitlich veränderliche Spannung & \si{\volt} \\
$\hat{u}$ & Amplitude der Spannung & \si{\volt} \\ 
$u_\mathrm{d}, u_\mathrm{q}$	&	Statorspannungs-Komponenten im d,q-System &	\si{\volt} \\
$u_\mathrm{ind}$	&	induzierte Spannung &	\si{\volt} \\
$i_\mathrm{d}, i_\mathrm{q}$	&	Statorstrom-Kompenenten im d,q-System	&	\si{\ampere} \\
$i_\mathrm{s}$	&	Ständerstrom	&	\si{\ampere} \\
$\hat{i}_\mathrm{s}$	&	Amplitude des Ständerstromes	&	\si{\ampere} \\

$\Theta$	&	magnetische Durchflutung 	&	\si{\ampere}	\\
$\Theta_\x{Nut}$	&	magnetische Durchflutung in der Nut 	&	\si{\ampere}	\\
$\Phi$		&	magnetischer Fluss			&	\si{\volt\second} \\
$\Psi$		&	verketteter Fluss			&	\si{\volt\second} \\
$\Psi_\mathrm{d}, \Psi_\mathrm{q}$	&	Flussverkettung im d,q-System	&	\si{\ampere} \\
$\Psi_\mathrm{pm}$		&	verketteter Fluss des Permanentmagneten			&	\si{\volt\second} \\
$x_\x{d}$	& 	realative synchron Reaktanz & \si{\ohm}\\

$\mu_\mathrm{0}$		&	magnetische Feldkonstante	&	\si{\volt\second\per\ampere\per\meter}\\
$\mu_\mathrm{r}$		&	relative Permeabilität		&	\\
$\epsilon_\x{RS}$ & realative Lage zwischen Ständer- und Rotorkoordinatensystem & \si{\radian} \\
$\epsilon_\x{0}$ & elektrische Feldkonstante & \si{\ampere\second\per\volt\per\meter} \\
$\epsilon_\x{1}$ & relative Permitivität & \si{\ampere\second\per\volt\meter} \\

$\rho$ & Raumladungsdichte & \si{\ampere\second\per\square\meter} \\
$\kappa$ & Leitfähigkeit & \si{\per\ohm\per\meter} \\
$\varphi$ & Phasenwinkel & \si{\radian} \\
$\tau_\i{p}$ & Polteilung & \si{\meter} \\
$\omega$ & Kreisfrequenz & \si{\per\second} \\
$\omega_\mathrm{el}$	&	elektr.\ Winkelgeschwindigkeit des Rotors & \si{\per\second} \\
$\omega_\mathrm{mech}$	&	mech.\ Winkelgeschwindigkeit des Rotors & \si{\per\second} \\
$\omega_\x{s}$ & Ständerkreisfrequenz & \si{\per\second} \\
\bottomrule
\end{tabular}
\end{flushleft}
% ----------------------------------------------------------------------------------------------------------
% Einleitung
% ----------------------------------------------------------------------------------------------------------
\pagenumbering{arabic}
\input{Einleitung}
% ----------------------------------------------------------------------------------------------------------
% Kapitel
% ----------------------------------------------------------------------------------------------------------
\cleardoublepage
% -*- coding: utf-8 -*-
% !TEX encoding = UTF-8 Unicode
% !TEX root =  main.tex

\chapter{Theoretische und begriffliche Grundlagen}\label{cha:grundlagen}

Um auf die Regelung einer anisotropen Synchronmaschine einzugehen, werden im folgenden einige Grundlagen erörtert.

\section{Theorie der Drehfeldmaschinen}\label{sec:grund-drehfeld}

Drehfeldmaschinen sind die am häufigsten eingesetzten Antriebsmaschinen, der Grund hierfür ist die Robustheit der aktiven Bauteile und die gute Energieeffizienz.
Zudem besitzen Drehfeldmaschinen ein großes Leistungsspektrum und einen großen Drehzahl- und Drehmomentstellbereich.
Die wesentlichen Vertreter der Maschinenfamilie sind die Synchron- und die Asynchronmaschinen.
Beide basieren auf der Wirkung eines Drehfeldes, das sich durch den Luftspalt der Maschine bewegt.
Die Synchron- und Asynchronmaschine besitzen im Ständer denselben Aufbau und erfordern zur Darstellung ihres Verhaltens eine Reihe gleicher physikalischer Begriffe.
Es ist zweckmäßig die Grundlagen der Synchronmaschine in einem eigenen Kapitel zu behandeln.
Dies gilt \insb für den Aufbau der Drehstromwicklungen sowie die Grundlagen zur Beschreibung von umlaufenden Durchflutungen und deren Felder.

\section{Magnetfelder}



\subsection{Strombelag}

Die örtliche und zeitliche Änderung von Magnetfeldern in elektrischen Maschinen wird durch die Anordnung der Leiter und die Art der Speisung bestimmt \autocite[S~199]{hofmann2013}.
Die räumliche Verteilung des Stromes wird durch den Strombelag bestimmt, der als $N\cdot I$ pro Länge des bewickelten Umfanges definiert ist.

Das Luftspaltfeld hat die zentrale Bedeutung und muss deshalb auch berechnet werden können.
Die Ursache für die Entstehung dieses Luftspaltfeldes sind die vom Strom durchflossenen Leiter in den Nuten des Stators.
Unter der idealisierten Annahme eines homogenen Feldverlaufs im Bereich der Nutöffnung (s.\ h.\ \autoref{fig:nutquerfeld}) das Feld im Luftspalt vom Feld in der Nut getrennt.

\begin{figure}[!htb]
\centering
\includegraphics[width=\textwidth]{nutquerfeld.pdf}
\label{fig:nutquerfeld}
\caption{Abbildung des Nutquerfeldes einer Rechtecknut im Stator.}
\end{figure}

Hierzu wird die oben abgebildete Nut betrachtet, wobei die Permeabilität des Eisens als sehr groß gegenüber derjenigen von Luft angenommen wird ($\mu_{Fe} \rightarrow \infty$).
Es bildet sich ein Nutquerfeld aus, das ist leicht aus dem Durchflutungsgesetzt herzuleiten.

\begin{align}
\oint \vec{H}d\vec{l} = \Theta \label{eqn:durchflutungsgesetzt}
\end{align}

Dieses Nutquerfeld stößt an der Grenzfläche zwischen Nutöffnung (Nutschlitz bzw.\ Streuschlitz) und Luftspalt an das zu berechnende Luftspaltfeld und stellt somit eine der Randbedingungen zur Berechnung des Luftspaltfeldes dar.
Das Magnetische Feld in der Nutöffnung $H_{S}$, dass unter idealisierte Annahme tangential gerichtet ist, kann wiederum auch aus dem Durchflutungsgesetz berechnet werden.

\begin{align}
H_S = \frac{\Theta_{Nut}}{b_S}
\end{align}

Diese Randbedingung zur Berechnung des Luftspaltfeldes kann auch anders erzeugt werden.
Unter Annahme, dass die Nutdurchflutung $\Theta$ unendlich dünn auf einer glatten Eisenoberfläche gleichmäßig im Bereich der Nutöffnung $b_S$ verteilt ist.
Diese Modellvorstellung wird mit Hilfe des Strombelages beschrieben.

\begin{align}
A = \frac{\Theta_{Nut}}{b_S}
\end{align}

\autoref{fig:strombelag-neu} zeigt die Modellvorstellung der obigen Beschreibung.

\begin{figure}[!h]
\centering
\includegraphics[width=\textwidth]{strombelag_neu.pdf}
\label{fig:strombelag-neu}
\caption{Vereinfachte Modellvorstellung zur Berechnung des Luftspaltfeldes mit Hilfe des Strombelags.}
\end{figure}

Bei Auswertung des Durchflutungsgesetz bei einem Umlauf um diesen Strombelag, ergibt sich für die tangentiale Feldstärke $H_t$ an der Eisenoberfläche im Bereich des Strombelages

\begin{align}
\oint \vec{H}d\vec{l} = \Theta_Nut \\
H_t\cdot b_S = A\cdot b_S \\
H_t = A = \frac{\Theta_{Nut}}{b_S} \label{eqn:feld-ht}
\end{align}

Mit \autoref{eqn:feld-ht} ist gezeit, dass die Randbedingungen zur Berechnung des Luftspaltfeldes unverändert erhalten bleibt, wenn statt der in Nuten eingebrachten Leiter ein äquivalenter Strombelag auf der glatten Eisenoberfläche berücksichtigt wird (Die Wirkung der Nutdurchflutung wird hinreichend genau durch den über der Nutöffnung verteilten Strombelag beschrieben).
Zur Berechnung des Luftspaltfeldes muss also nun das Nutenfeld \autocite[S.~14]{ternes2013} nicht berücksichtigt werden.
Zudem kann eine deutlich vereinfachte Geometrie zugrunde gelegt werden.

\begin{quote}
Die Begrenzungsflächen von Stator und Rotor können als glatt angenommen werden, was in Umfangsrichtung der Maschine konstanten Luftspalt und demzufolge auch einen kostanten magnetischen Luftspaltleitwert entspricht. (Gerling 2008, \emph{Elektrische Maschinen und Antriebe}. Bundeswehr Universität in München.)
\end{quote}

\subsection{Mehrphasensysteme}

Bei einpoliger Verbindung von $m$ Wechselspannungsquellen entsteht eine Schaltung, die $(m+1)$ Klemmen aufweist (s.~h.~\autoref{fig:mehrphasen}).
Haben diese $m$ Wechselspannungsquellen dieselbe Kreisfrequenz $\omega$, so stellt die Schaltung die Spannungsquelle eines allgemeinen Mehrphasensystems dar.

\begin{figure}[!h]
\centering
\includegraphics[width=\textwidth]{mehrphasen.pdf}
\label{fig:mehrphasen}
\caption{Spannungsquelle eines Mehrphasensystems.}
\end{figure}

Da keine Vorgaben bezüglich der Amplituden $\hat{u}$ und der Phasenlage $\varphi$ in der Definition der allgemeinen Mehrphasen-Spannungsquelle enthalten sind, kann sie \zB durch das folgende Gleichungssystemen beschrieben werden

\begin{align}
u_1(t) &= \hat{u}_1 \cdot \cos(\omega t - \varphi_1) \\
u_2(t) &= \hat{u}_2 \cdot \cos(\omega t - \varphi_2) \nonumber  \\
\vdots \nonumber \\
u_m(t) &= \hat{u}_m \cdot \cos(\omega t - \varphi_m) \nonumber
\end{align}

Aus der allgemeinen Mehrphasen-Spannungsquelle entsteht eine symmetrische Mehrphasen-Spannungsquelle, wenn zusätzlich gleiche Amplituden

\begin{align*}
\hat{u}_1 = \hat{u}_2 = \ldots \hat{u}_m
\end{align*}

und gleiche Phasenwinkeldifferenz zwischen aufeinanderfolgenden Teilspannungen gefordert werden

\begin{align*}
\varphi_1 - \varphi_2 = \varphi_2 - \varphi_3 = \ldots = \varphi_{m-1} - \varphi_m = \Delta \varphi
\end{align*}

Aus Symmetrieüberlegungen ergibt sich, dass die einheitliche Phasenwinkeldifferenz eine Funktion der Phasenzahl $m$ sein muss.

\begin{align}
\Delta \varphi = \frac{\omega T}{m} = \frac{2\pi}{m}
\end{align}

Darin tritt die Periodendauert $T$ der Teilspannungen auf.
Setzt man der Einfachheit

\begin{align*}
\varphi_1 = 0
\end{align*}

so wird die symmetrische Mehrphasen-Spannungsquelle durch das folgende Gleichungssystem beschrieben.

\begin{align}
u_1(t) &= \hat{u} \cdot \cos(\omega t) \\ \label{eqn:drehstrom}
u_2(t) &= \hat{u} \cdot \cos(\omega t - \frac{\omega T}{m})\nonumber \\
\vdots \nonumber \\
u_m(t) &= \hat{u} \cdot \cos(\omega t - (m-1)\frac{\omega T}{m})\nonumber
\end{align}

In der Elektrotechnik treten Systeme mit verschiedenen Phasenzahlen auf.
Das Wechselstromsystem kann als Sonderfall des Mehrphasensystems mit $m=1$ aufgefasst werden.
Es kommt nur bei kleinen Leistungen zum Einsatz.
Eine Ausnahme stellt die Bahnversorgung dar, die bis zu großen Leistungen generell einphasig betrieben wird.
Gekennzeichnet ist diese durch die eingeprägte Frequenz von $f = 16\frac{2}{3}\mbox{Hz}$.

Die Phasenzahl $m=2$ tritt bei elektrischen Kleinmaschinen auf, allerdings nur in Form eines unsymmetrischen Systems mit einer Phasenwinkeldifferenz

\begin{align*}
\Delta \varphi = 90^{\circ} \,\mbox{bzw.\ }\, 270^{\circ}
\end{align*}

Die Phasenzahl $m=3$ kennzeichnet das Drehstromsystem, dass die Basis der elektrischen Energietechnik bildet.
Höhere Phasenzahlen treten \zB in der Stromrichtertechnik auf mit $m=6, 12, 24$.
Drehstromerzeuger mit Phasenzahl $m=3$ werden generell als symmetrisches System ausgelegt.
Als Klemmenbezeichnung ist die Buchstabengruppe $R, S, T$ bzw.\ $U, V, W$ üblich, wobei die gemeinsame Leitung der drei Teilspannungen mit $O, N$ oder $Mp$ für Mittelpunkt bezeichnet wird.

Durch die DIN-Normung wurde festgelegt, dass die Klemmenbezeichnung beim Drehstromsystem mit $L1, L2$ und $L3$ zu erfolgen hat.
Die Phasenwinkeldifferenz ist $\Delta \varphi = 120^{\circ}$.
Stellt man die Phasenspannungen $u_1(t), u_2(t), u_3(t)$ nach \autoref{eqn:drehstrom} dar, so ergibt sich 

\begin{figure}[!h]
\centering
\includegraphics[width=\textwidth]{dreiphasensystem.pdf}
\label{fig:dreiphasen}
\caption{Phasenspannung eines symmetrischen Drehstromerzeugers.}
\end{figure}

Der prinzipielle Aufbau einer Drehstromwicklung lässt sich anhand aus den Anforderungen zur Erzeugung einer dreiphasigen Wechselspannung erläutern.
Eine solche Drehspannung erhält man mit einer Anordnung nach \autoref{fig:drehstromwicklung}.
Ein aus Dynamoblechen geschichtetes Ständerblechpacket enthält in Nuten am Bohrungsumfang gleichmäßig verteilte Leiter, die zu drei räumlich verteilten Wicklungssträngen zusammengeschaltet werden \autocite[S.~141]{fischer2009}.
Der Läufer erzeugt ein Gleichfeld, das eine sinusförmige Feldverteilung längst des Luftspaltes aufbaut.
Hat der Läufer eine konstante Drehzahl, so induziert das Feld in den einzelnen Spulen zeitlich sinusförmige Spannungen, die sich innerhalb eines Wicklungsstranges zu einem Wert addieren.
Die Berechnung der Induktion kann über die Allgemeine Beziehung

\begin{align}
u_q = B\cdot l \cdot v
\end{align}

erfolgen.
Sei $d_1$ der Bohrungdurchmesser des Ständerblechpaketes einer $2p$-poligen Maschine, so bezeichnet man den Umfangsanteil

\begin{align}
\tau_p = \frac{d_1 \cdot \pi}{2p}
\end{align}

wieder als Polteilung.
Die Polteilung entspricht der Länder einer Halbwelle der sinusförmigen Flussdichteverteilung im Luftspalt (enspricht einem elektrischen Winkel von $\omega t = 180^{\circ}$.
Bei einer zweipoligen Maschine mit $p=1$ stimmen somit der räumlich mechanische und der elektrische Winkel überein, allgemein gilt die Beziehung \autocite[S.141f.]{fischer2009}

\begin{align}
\gamma_{el} = p\cdot \gamma_{mech}
\end{align}

BBB Noch überarbeiten, Quellen: Fischer, Schröder, DISS von Bruhn BBB

BBB Abbildungen erstellen BBB

BBB Formeln? BBB

BBB Nicht äquivalent??? BBB

\begin{figure}[!htb]
\centering
\includegraphics[width=\textwidth]{synchronmaschine-drehstrom.pdf}
\label{fig:drehstromwicklung}
\caption{Erzeugung einer mehrphasigen Spannung durch ein räumlich sinusförmiges Läuferdrehfeld.}
\end{figure}

\section{Induktivitäten}\label{sec:induktiv}

\section{Einführung Synchronmaschine}\label{sec:synchron}

\paragraph{Historischer Hintergund} Die ersten Synchronmaschinen wurden als Einphasengenerator entwickelt und gebaut, den ersten dreiphasigen Synchrongenerator entwickelten 1887 unabhängig voneinander F.~A.~Haselwander\footnote{Friedrich August Haselwander war ein deutscher Ingenieur, ein Erfinder der Drehstrom-Synchronmaschine und des kompressorlosen Ölmotors.} und C.~S.~Bradley\footnote{Charles Schenk Bradley war ein US-amerikanischer Elektrotechniker, Erfinder und Pionier von frühen Elektromotoren. Er zählt neben F.~A.~Haselwander zu den Begründern des heute im Bereich der elektrischen Energietechnik eingesetzten Dreiphasenwechselstromes.} Bei den Entwicklungen bildeten sich die Bauformen der Schenkelpol- und Vollpolmaschine aus. Die Weiterentwicklung der Synchronmaschine hing stark mit dem Ausbau der Energieversorgung und dem Bedarf von leistungsstärkeren Generatoren zusammen. Unabhängig von der Entwicklung wurden schon sehr früh Synchronmaschinen als Antriebsmaschinen für eine konstante Drehzahlregelung oder einen Phasenbetrieb in der Industrie eingesetzt \autocites[S.~287]{fischer2009}[S.~485f.]{mullerI2005}.

Die gleichstromgespeiste Erregerwicklung ermöglicht es, das Magnetfeld unabhängig vom Netz zu beeinflussen.
Als Spannungsquelle für die Speisung der Erregerwicklung wurden sog.\ Gleichstromerregermaschinen eingesetzt, in der heutigen Zeit werden Wechselspannung mit Hilfe von Leistungselektronischen Schaltungen gespeist.
Um die Schleifringübertragung der Erregerleistung zu umgehen, werden schleifring- bzw.\ bürstenlose Erregersysteme realisiert \autocite[S.~108]{ternes2012}.
Als Motor wurden Drephasen-Synchronmaschinen schon bald für große Leistungen eingesetzt, \zB zum Antrieb von Pumpen und Verdichten \autocite[S.~486]{mullerI2005}.
Der Nachteil ist, dass die Drehzahl durch die Netzfrequenz festgelegt ist.
Die Synchronmaschine arbeitet unabhängig von der Belastung stets mit der durch die Netzfrequenz und die ausgeführte Polpaarzahl festgelegten synchronen Drehzahl.

Heute ist es möglich mit Hilfe eines Frequenzumrichters die Drehzahl der Synchronmaschine zu steuern.
Aus diesem Grund werden größere Gleichstrommaschinen durch drehzahlvariable Synchronmaschinen abgelöst.
Im Bereich kleinerer Leistungen wird anstelle der Gleichstromerregung eine Erregung durch Permanentmagnete eingesetzt.
Dabei verliert man die Beeinflussung des Erregerzustandes über den Erregerstrom, dafür erhält man eine elektrische Maschine die keine elektrische Verbindung zum Läufer erfordert.

\subsection{Spannungsgleichungen und Ersatzschaltbild}

Die Synchronmaschine mit Vollpolläufer ist wegen ihres konstanten Luftspaltes mathematisch leichter erfassbar, als die Synchronmaschine mit Schenkelpolläufer.
Als Grundlage für weitere Betrachtungen dient dieses mathematische Modell als Grundlage.
Weiterhin wird vereinbart, dass
\begin{itemize}
	\item quasistationärer Betrieb
	\item Verbraucherzählpfeilsystem
	\item rechtsgängige Spulen
	\item läuferfeste, komplexe Ebene
\end{itemize}

\begin{figure}[!h]
\centering
\includegraphics[width=\textwidth]{synchron-dq-1.pdf}
\label{fig:dq-synchron-1}
\caption{d,q-System der Synchronmaschine als Skizze.}
\end{figure}

\subsection{Beschreibung der Synchronmaschine im dq-Koordinatensystem}

Im folgenden wird angenommen, dass die Speisung des Polrads durch Permanentmagneten ersetzt wird.
In diesem Fall verbleiben nur die drei Statorwicklungen als stromdurchflossene Wicklungen.
Wesentlich bei den nachstehenden Überlegungen ist es, ob die Synchronmaschine als symmetrische Maschine (Vollpolläufer) oder als unsymmetrische Maschine (Schenkelpolläufer) konzipiert ist.
Die Wahl der Konzipierung hat Auswirkungen auf die Möglichkeit, Feldschwächebetrieb zu erreichen oder nur bedingt und dann mit Einschränkungen \autocite[S.~291]{schroder2000}.
Wird die Synchronmaschine in der Statorwicklung mit einer sinusförmigen Spannung versorgt, so ist diese als (PMSM) permanentmagneterregte Synchronmaschine definiert.
Bei einer trapezförmigen Speisung der Statorwicklung wird die Maschine als (BLDC) bürstenlose Gleichstrommaschine bezeichnet.
Der einfachste Fall für die Ermittlung des Signalflussplanes ist die Annahme, dass die Maschine an der Statorwicklung eine sinusförmige Spannung anliegt und die Maschine symmetrisch konzipiert wurde.
Bei einer symmetrisch konzipierten Synchronmaschine werden die Reluktanzeinflüsse nicht wirksam.
Damit ergeben sich nach \autocite[S.~291]{schroder2000} folgende Gleichungen für die Synchron-Vollpolmaschine.

\begin{align}
\frac{d\psi_d}{dt} &= U_d - R_1\cdot I_d + \Omega_L\cdot\psi_q \\
\frac{d\psi_q}{dt} &= U_q - R_1\cdot I_q + \Omega_L\cdot\psi_d \\
\end{align}

mit 

\begin{align}
\psi_d &= \psi_{PM} + L_d\cdot I_d\\
\psi_q &= L_q\cdot I_q
\end{align}

Damit ergibt sich das innere Luftspaltdrehmoment $M_i$ zu:

\begin{align}
M_i &= \frac{3}{2}\cdot Z_p \cdot (\psi_d\cdot I_q - \psi_q\cdot I_d) \\
\nonumber &= \frac{3}{2}\cdot Z_p \cdot (\underbrace{\psi_{PM}\cdot I_q}_{\text{~Hauptmoment}} + \underbrace{(L_d-L_q)\cdot I_d\cdot I_q}_{\text{~Reluktanzmoment}})
\end{align}

und der allgemeinen mech.\ Bewegungsgleichung

\begin{align}
\Theta\cdot \frac{d\Omega_m}{dt} = M_i - M_L
\end{align}

\begin{figure}[!htb]
\centering
\includegraphics[width=\textwidth]{synchron-dq.pdf}
\label{fig:synchron-dq}
\caption{Darstellung der Synchronmaschine im dq-Koordinatensystem.}
\end{figure}

\section{Besonderheiten der Schenkelpolmaschine}\label{sec:schenkelpol}

\section{Permanenterregte Synchronmaschine}\label{sec:pmsm}

\section{Evalurierung der Ersatzschaltbilder für die Regelung}\label{sec:esb}

\begin{figure}[!htb]
\centering
\includegraphics[width=\textwidth]{synchron-esb-kremser2004.pdf}
\label{fig:esb-kremser}
\caption{Ersatzschaltbild der Synchronmaschine nach \parencite[S.~145]{kremser2004}.}
\end{figure}

%%% Local Variables: 
%%% mode: latex
%%% TeX-master: "main"
%%% TeX-open-quote: "\\enquote{"
%%% TeX-close-quote: "}"
%%% LaTeX-csquotes-open-quote: "\\enquote{"
%%% LaTeX-csquotes-close-quote: "}"
%%% End: 


% -*- coding: utf-8 -*-
% !TEX encoding = UTF-8 Unicode
% !TEX root =  main.tex

\chapter{Grundlagen der Vektorregelung}
\label{cha:Grundlagen der Vektorregelung}

%% -- Einführung in die Vektorregelung
In modernen Antriebssystemen ist es häufig unerlässlich, entscheidende Maschinengrößen wie Drehzahl oder Drehmoment auf einen gewünschten Wert einzustellen.
Dabei kamen der Vergangenheit häufig Gleichstrommaschinen zum Einsatz, welche sich durch eine gute Regel- und Einstelleigenschaften bei den geforderten Parametern auszeichnen.
Große Fortschritte in den Bereichen der Leistungselektronik und bei Regelkomponenten führen dazu, dass heute Antriebe, ohne besonderen Aufwand, mit Synchronmaschinen realisiert werden können.
Gleichzeitig haben die Drehfeldmaschinen den Vorteil, dass Aufgrund der fehlenden mechanischen Kommutation kein nennenswerter Verschleiß Auftritt.

Entscheidend für den Aufbau einer geregelten PMSM ist die Vektor- bzw.\ feldorientierte Regelung. 
Die Maschine wird mit näherungsweise sinusförmig veränderlichen Strömen gespeist. 
Ebenso besitzen alle weiteren auftretenden elektrischen Größen wie Spannungen, Flüsse oder Felder aufgrund ihres Zeitverhaltens annähernd Sinusform \parencite[S.~1]{nuss2010}.	
Die Idee der Vektorregelung ist es nun, nicht die zeitlichen Momentanwerte der Ströme zu verändern, sondern die erfassten Wechselgrößen in ein Zwei-komponentiges rotierendes Koordinatensystem zu übertragen.
Diese Komponenten werden regelungstechnisch verwertet und zurück transformiert.

\section{Raumzeigerdarstellung}
\label{sec:raumzeiger}

Die stationären Zusammenhänge der elektrischen Größen in der Maschine, welche ursächlich aus dem Zusammenhang von $\Psi$ und B herrühren, können zunächst mithilfe komplexer Zeitzeiger beschrieben werden. Dabei lassen sich die Statorströme, $i_{s,1}$, $i_{s,2}$, und $i_{s,3}$ einer Drehfeldmaschine mit idendischer Amplitunde $\hat i_{s}$ und Statorkreisfrequenz $\omega_{s}$  und einer jeweiligen $120^\circ$ Phasenverschiebung als

\begin{align}
	\begin{split}
	i_{s,i} = Re\{\underline{i}_{s,i}\} = Re\{\underline{\hat i}_{s,i}\cdot e^{j\omega_{s}t}\} = Re\Bigl\{\hat{i_{s}}\cdot e^{j(\omega_{s}t+0-(i-1)\cdot\tfrac{2\pi}{3})}\Bigr\}
	\\= \hat{i}_{s}\cdot cos\bigg(\omega_{s}t+\varphi_{0}-(i-1)\cdot\frac{2\pi}{3}\bigg)~;~mit~i=1,2,3 \label{statorströme} 
\end{split}
\end{align}

mit den komplexen Zeitzeigern

\begin{align}
	\underline{i}_{s,i} = \underline{\hat i}_{s,i}\cdot e^{j\omega_{s}t}~;~mit~i=1,2,3 \label{zeitzeiger}
\end{align}

und den komplexen Amplituden

\begin{align}
	\underline{\hat i}_{s,i} = \hat{i_{s}}\cdot e^{j(\omega_{s}t+0-(i-1)\cdot\tfrac{2\pi}{3})}~;~mit~i=1,2,3 \label{amplituden}
\end{align}

darstellen. 
Die folgende Abbildung \ref{fig:zeitzeiger} veranschaulicht die vorangegangenen Gleichungen \ref{statorströme}, \ref{zeitzeiger} sowie \ref{amplituden} und stellt beispielhaft den Zeitzeiger $i_{s,1}$ dar.

\begin{figure}[h]
	\centering
	\includegraphics[width=0.5\textwidth]{zeitzeiger.pdf}
	\label{fig:zeitzeiger}
	\caption{Beispielhafte Lage eines Zeitzeigers.}
\end{figure}

Da das Ziel darin besteht, den dynamischen Rotationsvorgang einer PMSM zu modellieren, ist die Verwendung eines Zeitzeigers, mit dem nur stationöre Vorgänge beschrieben werden können, nicht angebracht. 
Hier ist es Zweckmäßig, einen Operator so zu entwickeln, dass dieser in der Lage ist, dynamische Vorgänge zu beschrieben, ohne dazu Nebenbedingungen wie beispielsweise die Periodizität heranzuziehen. 
Bei der Entwicklung bieten sich die Statorphasenströme $i_{s,1}$, $i_{s,2}$, und $i_{s,3}$ des Dreiphasensystems an.
Diese stehen zu jedem Zeitpunkt zur Verfügung. 
Es sei angemerkt, dass dabei die Nullbedingung erfüllt ist. 
Die Summe der Statorphasenströme muss immer null sein, was beim Einsatz von Drehfeldmaschinen idr. gegeben ist.
Dadrurch ist es auch immer möglich mit Kenntnis zweier Größen auf die dritte zu schließen.
Nun ist zweikomponentige Zeitzeiger immer um mindestens zwei Momentanwerte erweiterbar. 
Ein hierfür geeigneter Ansatz zur Erzeugung eines Raumzeigers wurde erstmals in kovacs1959 veröffentlicht:

\begin{align}
	\underline{i}_{s}(t) = \frac{2}{3} \cdot \Bigl\{\underline{i}_{s,1}(t) + \underline{i}_{s,2}(t)\cdot e^{j\tfrac{2\pi}{3}} + \underline{i}_{s,3}(t)\cdot e^{j\tfrac{4\pi}{3}} \Bigr\} \label{statorstromzeiger}
\end{align}

Um jetz aufzeigen zu können, dass der Ansatz aus \ref{statorstromzeiger} im stationären Zustand mit dem entsprechenden Statorstromzeitzeiger übereinstimmt und schlussendlich den Raumzeiger zu erzeugen, werden zunächst in \ref{statorstromzeiger} die Statortrommomentanwerte aus \ref{statorströme} eingesetzt.
Dadurch erhält man

\begin{align}
\underline{i}_{s}(t) = \frac{2}{3} \cdot \biggl\{\hat{i}_{s}\cdot cos(\omega_{s}t+\varphi_{0}) + \hat{i}_{s}\cdot cos\bigg(\omega_{s}t+\varphi_{0}-\frac{2\pi}{3}\bigg)\cdot e^{j\tfrac{2\pi}{3}} + \hat{i}_{s}\cdot cos\bigg(\omega_{s}t+\varphi_{0}-\frac{4\pi}{3}\bigg)\cdot e^{j\tfrac{4\pi}{3}} \biggr\}
\label{zwischen1}
\end{align}

Wird nun die trigonometrische Cosinus-Funktion durch die entsprechende exponentielle Darstellung ersetzt, folgt hieraus

\begin{align}
	\begin{split}
	\underline{i}_{s}(t) = \frac{2}{3} \cdot \hat{i}_{s} \cdot \biggl\{ \frac{1}{2} \cdot ( e^{j(\omega_{s}t+\varphi_{0})} + e^{-j(\omega_{s}t+\varphi_{0})} ) + \frac{1}{2} \cdot ( e^{j(\omega_{s}t+\varphi_{0}-\frac{2\pi}{3})} + e^{-j(\omega_{s}t+\varphi_{0}-\frac{2\pi}{3})})\cdot e^{j\tfrac{2\pi}{3}} + \\ \frac{1}{2} \cdot ( e^{j(\omega_{s}t+\varphi_{0}-\frac{4\pi}{3})} + e^{-j(\omega_{s}t+\varphi_{0}-\frac{4\pi}{3})})\cdot e^{j\tfrac{4\pi}{3}}  \biggr\}
	\label{zwischen2}
	\end{split}
\end{align}

Nach ausmultiplizieren der Therme folgt mit $1+e^{j\tfrac{4\pi}{3}}+e^{j\tfrac{8\pi}{3}}=0$ das Ergebnis und somit der Raumzeiger

\begin{align}
	\underline{i}_{s} = \frac{2}{3} \cdot \hat{i}_{s}\cdot\biggl\{ \frac{3}{2} \cdot  e^{j(\omega_{s}t+\varphi_{0})} + \frac{1}{2} \cdot e^{-j(\omega_{s}t+\varphi_{0})} \cdot \bigg( 1+e^{j\tfrac{4\pi}{3}}+e^{j\tfrac{8\pi}{3}} \bigg)  \biggr\} = \hat{i}_{s} \cdot e^{j(\omega_{s}t+\varphi_{0})}
	\label{zeigerende}
\end{align}

Das Ergebnis von \ref{zeigerende} entspricht strukturell dem in \ref{statorströme} angegebenen Statorstromzeitzeiger. Dadurch ist sichergestellt, dass der Ansatz aus \ref{statorstromzeiger} in der Lage ist als Gesamtzeiger, bestehend aus den Momentanwerten der Statorströme, zu fungieren. 
Die folgende Abbildung \ref{fig:raumzeigermaschine} zeigt zur Veranschaulichung eine zweipolige Drehfeldmaschine mit zugehörigem Zeigerdiagramm, welches den Statorstromraumzeiger beinhaltet. 

\begin{figure}[h]
	\centering
	\includegraphics[width=0.5\textwidth]{raumzeigermaschine.pdf}
	\label{fig:raumzeigermaschine}
	\caption{zweipolige Drehfeldmachine mit beispielhafter Statorstromraumzeigerlage}
\end{figure}

Mit der Einführung des Raumzeigers ist die theoretische Grundlage dafür geschaffen, die PMSM mit einer feldorientierten Regelung zu versehen. 
Da sich wie Eingangs beschrieben alls Größen in der Drehfeldmaschine näherungsweise sinusförmig verhalten, ist die Stromraumzeigerdarstellung aus \ref{statorstromzeiger} für alle andren dreiphasigen Größen als allgemeine Raumzeigerdarstellung definierbar.

\begin{align}
	\underline{a}(t) = \frac{2}{3} \cdot \Bigl\{\underline{a}_{1}(t) + \underline{a}_{2}(t)\cdot e^{j\tfrac{2\pi}{3}} + \underline{a}_{3}(t)\cdot e^{j\tfrac{4\pi}{3}} \Bigr\} \label{raumzeigerdefinition}
\end{align}

Im Folgenden werden die in der Praxis benötigten Transformationsvorschriften erläutert, welche das Wechseln zwischen Phasen- und Raumzeigergrößen erlauben.


\section{Beschreibung in $\alpha$-$\beta$-Koordinatensystem}\label{sec:clark}

Als Grundlage für das Wechseln zwischen Phasen- und Raumzeigergrößen dienz zunächst die Definition aus \ref{raumzeigerdefinition}. Die Definitionsgleichung lässt sich in Real- und Imaginärteil aufspalten. Es kommt so zu folgender Aufteilung

\begin{align}
	Re{\underline{a}(t)} =
	\label{realteil}
\end{align}

\begin{align}
	\label{imaginärteil}
\end{align}

\section{Beschreibung in rotorfesten d-q-Koordinatensystem}\label{sec:park}

%\section{Einführung in die Raumzeigermodulation}\label{sec:raumzeiger}

%\subsection{Begriff des Raumzeigers}

\subsection{Transformation zwischen Phasen- und Raumzeigergrößen}

\subsection{Raumzeigertransformation zwischen ortsfesten und rotierenden Bezugssystemen}

\section{Signalflussplan der Vektorregelung}\label{sec:signalflussplan}



%%% Local Variables: 
%%% mode: latex
%%% TeX-master: "main"
%%% TeX-open-quote: "\\enquote{"
%%% TeX-close-quote: "}"
%%% LaTeX-csquotes-open-quote: "\\enquote{"
%%% LaTeX-csquotes-close-quote: "}"
%%% End: 

% -*- coding: utf-8 -*-
% !TEX encoding = UTF-8 Unicode
% !TEX root =  main.tex

\chapter{Regelung der PMSM mit Simulink}
\label{cha:regelungpmsm}

\section{Einführung in Simulink}\label{sec:simulink}

\section{Einführung in die TI-Bibliotheken}\label{sec:TI}

\section{Übersicht der Regelstruktur}\label{sec:ueberregelung}

\subsection{Subsysteme}





%%% Local Variables: 
%%% mode: latex
%%% TeX-master: "main"
%%% TeX-open-quote: "\\enquote{"
%%% TeX-close-quote: "}"
%%% LaTeX-csquotes-open-quote: "\\enquote{"
%%% LaTeX-csquotes-close-quote: "}"
%%% End: 

% -*- coding: utf-8 -*-
% !TEX encoding = UTF-8 Unicode
% !TEX root =  main.tex

\chapter{Zusammenfassung}
\label{cha:zusammenfassung}

Die Zielsetzung des Projektes ist die vollständige Implementierung einer Vektorregelung für Synchronmaschinen in \product{Matlab/Simulink} unter Verwendung der Bibliotheken für Texas Instruments DSPs.
Zu den Aufgabenstellungen die erfüllt wurden, gehört eine ausführliche Literaturrecherche zum Maschinenmodell einer anisotropen Synchronmaschine und die Implementierung des mathematischen Modells in \product{Simulink}.
Im Rahmen des Projektes wurde eine ausführliche und übersichtliche Dokumentation zu den oben aufgeführten Aufgaben erstellt.
Beiliegend eine CD mit dem Maschinenmodell und den benutzten \product{Simulink}-Bausteinen.
%Der Vergleich mit der Texas Instrument Bibliothek gestaltete sich als schwierig und sehr aufwändig.
%Die Lauffähigkeit der Modelle stellte sich als nicht stabil heraus, da Änderungen an den Parametern zu unerwünschten Effekten führte.
Die Lauffähigkeit der Modelle ist nicht gewährleistet, weil zum einen die synchrone Drehzahl bei der Simulation mit starren Netz nicht erreicht wurde (s.~h.~Abschnitt~\ref{sec:sim-starr-netz}) und zum anderen führten Parameteränderungen zu unerwünschten Nebeneffekten (z.\~B.\ die Drehzahl ist während des Einschwingvorganges negativ geworden).
Durch das Vernachlässigen der Sättigung der Blechpackete bei den Modellen werden die Ströme während des Einschwingvorganges sehr hoch (s.~h.~Abbildung~\ref{fig:drehmoment}).
Die Clarke-Park-Transformation ist bei dem Modell mit starren Netz nicht wie erwartet gleichförmig gewesen (s.~h.~Abbildung~\ref{fig:n-netz}).
Bestehende Modelle der TI-Bibliothek sind nicht zum Vergleich geeignet gewesen, da diese mit einer Raumzeigermodulation modelliert sind und somit nicht zum Vergleich mit dem starren Netz möglich waren.
Außerdem waren viele Modelle mit der Bibliothek von Simscape ausgestattet, sodass ein Vergleich nicht sinnvoll war.
Das Modell mit der PWM funktioniert entspricht den Anforderungen an Funktionalität.

Für zukünftige Arbeiten in diesem Bereich empfiehlt sich eine ausführliche Literaturrecherche für die TI-Bausteine und die Implementierung von \enquote{Saturation-Blocks}.
Die Sättigung der Blechpackete scheint von größerer Bedeutung zu sein, als angenommen.
Zusätzlich ist es notwendig, die Parameter der Regelung besser einzustellen, da diese einen sehr großen Einfluss auf die dargestellten Ergebnisse im Kapitel~\ref{chap:ergebnisse-foc} haben.
Die Problematik mit der Clarke-Park-Transformation stellt sich als ungelöst dar und muss entsprechend korrigiert werden, damit das Modell funktionsfähig wird.
Zusätzlich bietet es sich an, andere Simulationsverfahren zu verwenden, wie z.\ B.\ Verfahren mit variabler Schrittweite.



%%% Local Variables: 
%%% mode: latex
%%% TeX-master: "main"
%%% TeX-open-quote: "\\enquote{"
%%% TeX-close-quote: "}"
%%% LaTeX-csquotes-open-quote: "\\enquote{"
%%% LaTeX-csquotes-close-quote: "}"
%%% End: 

\listoffigures
% ----------------------------------------------------------------------------------------------------------
% Literatur
% ----------------------------------------------------------------------------------------------------------
\cleardoublepage
\nocite{*}
\printbibliography
% ----------------------------------------------------------------------------------------------------------
% Anhang
% ----------------------------------------------------------------------------------------------------------
\begin{appendix}
\chapter{Simulationsblöcke}
\section{Elektrische Komponenten~--~Subsysteme}
\begin{figure}[htb]
	\centering
	\includegraphics[width=0.8\textwidth]{/simulink/electrical-system.pdf}
	\label{fig:electrical-system}
	\caption{Aufbau des elektrischen Subsystemsystems.}
\end{figure}

\begin{figure}[htb]
\centering
\includegraphics[width=\textwidth]{/simulink/mechanical-system.pdf}
\label{fig:mechanical-system}
\caption{Aufbau des mechanischen Subsystems.}
\end{figure}
\newpage
\section{Transformations Subsysteme}
\begin{figure}[h]
	\centering
	\includegraphics[scale=0.7]{/simulink/uvw_to_ab.pdf}
	\label{fig:uvw_to_ab}
	\caption{Aufbau der Clarke-Transformation.}
\end{figure}

\begin{figure}[h]
	\centering
	\includegraphics[scale=0.7]{/simulink/ab_to_dq.pdf}
	\label{fig:uvw_to_ab}
	\caption{Aufbau der Park-Transformation.}
\end{figure}


	
\end{appendix}

\end{document}