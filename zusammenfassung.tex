% -*- coding: utf-8 -*-
% !TEX encoding = UTF-8 Unicode
% !TEX root =  main.tex

\chapter{Zusammenfassung}
\label{cha:zusammenfassung}

Die Zielsetzung des Projektes ist die vollständige Implementierung einer Vektorregelung für Synchronmaschinen in Matlab Simulink unter Verwendung der Bibliotheken für Texas Instruments DSPs.
Zu den Aufgabenstellungen die erfüllt werden sollen gehört eine ausführliche Literaturrecherche zum Maschinenmodell einer anisotropen Synchronmaschine und die Implementierung des mathematischen Modells in Simulink.
Im Rahmen des Projektes wurde eine ausführliche und übersichtliche Dokumentation zu den oben aufgeführten Aufgaben erstellt.
Beiliegend eine CD mit dem Maschinenmodell und benutzten Simulink Bausteinen.
Der Vergleich mit der Texas Instrument Bibliothek gestaltete sich als schwierig und sehr aufwändig.
Die Lauffähigkeit der Modelle stellte sich als nicht stabil heraus, da Änderungen an den Parametern zu unerwünschten Effekten führte.
Bestehende Modelle der TI-Bibliothek sind nicht zum Vergleich geeignet gewesen, da diese mit einer Raumzeigermodulation modelliert sind und somit nicht zum Vergleich mit dem Starren Netz möglich waren.






%%% Local Variables: 
%%% mode: latex
%%% TeX-master: "main"
%%% TeX-open-quote: "\\enquote{"
%%% TeX-close-quote: "}"
%%% LaTeX-csquotes-open-quote: "\\enquote{"
%%% LaTeX-csquotes-close-quote: "}"
%%% End: 
