% -*- coding: utf-8 -*-
% !TEX encoding = UTF-8 Unicode
% !TEX root =  main.tex

\chapter{Zusammenfassung}
\label{cha:zusammenfassung}

Die Zielsetzung des Projektes ist die vollständige Implementierung einer Vektorregelung für Synchronmaschinen in \product{Matlab/Simulink} unter Verwendung der Bibliotheken für Texas Instruments DSPs.
Zu den Aufgabenstellungen die erfüllt wurden, gehört eine ausführliche Literaturrecherche zum Maschinenmodell einer anisotropen Synchronmaschine und die Implementierung des mathematischen Modells in \product{Simulink}.
Im Rahmen des Projektes wurde eine ausführliche und übersichtliche Dokumentation zu den oben aufgeführten Aufgaben erstellt.
Beiliegend eine CD mit dem Maschinenmodell und den benutzten \product{Simulink}-Bausteinen.
%Der Vergleich mit der Texas Instrument Bibliothek gestaltete sich als schwierig und sehr aufwändig.
%Die Lauffähigkeit der Modelle stellte sich als nicht stabil heraus, da Änderungen an den Parametern zu unerwünschten Effekten führte.
Die Lauffähigkeit der Modelle ist nicht gewährleistet, weil zum einen die synchrone Drehzahl bei der Simulation mit starren Netz nicht erreicht wurde (s.~h.~Abschnitt~\ref{sec:sim-starr-netz}) und zum anderen führten Parameteränderungen zu unerwünschten Nebeneffekten (z.\~B.\ die Drehzahl ist während des Einschwingvorganges negativ geworden).
Durch das Vernachlässigen der Sättigung der Blechpackete bei den Modellen werden die Ströme während des Einschwingvorganges sehr hoch (s.~h.~Abbildung~\ref{fig:drehmoment}).
Die Clarke-Park-Transformation ist bei dem Modell mit starren Netz nicht wie erwartet gleichförmig gewesen (s.~h.~Abbildung~\ref{fig:n-netz}).
Bestehende Modelle der TI-Bibliothek sind nicht zum Vergleich geeignet gewesen, da diese mit einer Raumzeigermodulation modelliert sind und somit nicht zum Vergleich mit dem starren Netz möglich waren.
Außerdem waren viele Modelle mit der Bibliothek von Simscape ausgestattet, sodass ein Vergleich nicht sinnvoll war.
Das Modell mit der PWM funktioniert entspricht den Anforderungen an Funktionalität.

Für zukünftige Arbeiten in diesem Bereich empfiehlt sich eine ausführliche Literaturrecherche für die TI-Bausteine und die Implementierung von \enquote{Saturation-Blocks}.
Die Sättigung der Blechpackete scheint von größerer Bedeutung zu sein, als angenommen.
Zusätzlich ist es notwendig, die Parameter der Regelung besser einzustellen, da diese einen sehr großen Einfluss auf die dargestellten Ergebnisse im Kapitel~\ref{chap:ergebnisse-foc} haben.
Die Problematik mit der Clarke-Park-Transformation stellt sich als ungelöst dar und muss entsprechend korrigiert werden, damit das Modell funktionsfähig wird.
Zusätzlich bietet es sich an, andere Simulationsverfahren zu verwenden, wie z.\ B.\ Verfahren mit variabler Schrittweite.



%%% Local Variables: 
%%% mode: latex
%%% TeX-master: "main"
%%% TeX-open-quote: "\\enquote{"
%%% TeX-close-quote: "}"
%%% LaTeX-csquotes-open-quote: "\\enquote{"
%%% LaTeX-csquotes-close-quote: "}"
%%% End: 
