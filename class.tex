\documentclass[fontsize=12pt,%
paper=a4,%
DIV=classic,%
BCOR=8mm,%
twoside=false,%
headings=openany,%
parskip=half,%
pagesize=auto,%
numbers=noenddot,%
headsepline=true,%
toc=listof,%
toc=bibliography%
]{scrreprt}

% PDF-Kompression
\pdfminorversion=5
\pdfobjcompresslevel=1

% Schriften
\usepackage{mathpazo,tgpagella}
%\usepackage{libertine}
%\usepackage{fourier}
\usepackage{lmodern}

% Allgemeines
\usepackage{amsmath,amssymb} % Mathesachen
\usepackage[T1]{fontenc} % Ligaturen, richtige Umlaute im PDF
\usepackage[utf8]{inputenc}% UTF8-Kodierung für Umlaute usw

\usepackage{siunitx}
\usepackage{scrlayer-scrpage} 
\pagestyle{scrheadings}


%\usepackage{setspace}
%\onehalfspacing 

% Schriften-Größen
\setkomafont{chapter}{\Huge\rmfamily}
\setkomafont{section}{\Large\rmfamily}
\setkomafont{subsection}{\large\rmfamily}
\setkomafont{subsubsection}{\large\rmfamily}
\setkomafont{chapterentry}{\large\rmfamily} % Überschrift der Ebene in Inhaltsverzeichnis
\setkomafont{descriptionlabel}{\bfseries\rmfamily} % für description Umgebungen
\setkomafont{captionlabel}{\small\bfseries}
\setkomafont{caption}{\small}

% Sprache: Deutsch
\usepackage[autostyle,babel,german=guillemets,style=german]{csquotes}
\usepackage[USenglish,ngerman]{babel} 
\selectlanguage{ngerman}
% PDF
\usepackage[ngerman,%
pdfauthor={B. Ternes},%
pdftitle={Modellbasierte Implementierung einer Vektorregelung für Synchronmaschinen},%
colorlinks=true,linkcolor=blue,citecolor=blue,filecolor=magenta,urlcolor=blue%
]{hyperref}

% BibLaTeX
\usepackage[backend=biber,%
 style=authoryear,%
 autocite=inline,%
 sorting=anyt,%
 sortcites=true,%
 hyperref=auto,%
 maxnames=2,%
 minnames=1,%
]{biblatex}

\bibliography{bibo.bib}

\setlength{\bibitemsep}{1em}
\DefineBibliographyStrings{ngerman}{andothers = {u.\,a.},}

\usepackage[final]{microtype} % mikrotypographische Optimierungen
\usepackage{url}
\usepackage{pdflscape} % einzelne Seiten drehen können

% Tabellen
\usepackage{multirow} % Tabellen-Zellen über mehrere Zeilen
\usepackage{multicol} % mehre Spalten auf eine Seite
\usepackage{tabularx} % Für Tabellen mit vorgegeben Größen
\usepackage{longtable} % Tabellen über mehrere Seiten
\usepackage{array}

%  Bibliographie
%\usepackage{bibgerm} % Umlaute in BibTeX

% Tabellen
\usepackage{multirow} % Tabellen-Zellen über mehrere Zeilen
\usepackage{multicol} % mehre Spalten auf eine Seite
\usepackage{tabularx} % Für Tabellen mit vorgegeben Größen
\usepackage{array}
\usepackage{float}

% Bilder
\usepackage{graphicx}
\usepackage{color}
\graphicspath{{_Bilder/}}

\DeclareGraphicsExtensions{.pdf,.png,.jpg}
\usepackage{subfigure}
\newcommand{\subfigureautorefname}{\figurename}
\usepackage[all]{hypcap}

% Bildunterschrift
\setcapindent{0em}
\setcapwidth[c]{0.9\textwidth}
\setlength{\abovecaptionskip}{0.2cm}

% Quellcode
\usepackage{listings}
\definecolor{grau}{gray}{0.25}
\lstset{
	extendedchars=true,
	basicstyle=\tiny\ttfamily,
	%basicstyle=\footnotesize\ttfamily,
	tabsize=2,
	keywordstyle=\textbf,
	commentstyle=\color{grau},
	stringstyle=\textit,
	numbers=left,
	numberstyle=\tiny,
	% für schönen Zeilenumbruch
	breakautoindent  = true,
	breakindent      = 2em,
	breaklines       = true,
	postbreak        = ,
	prebreak         = \raisebox{-.8ex}[0ex][0ex]{\Righttorque},
}

% linksbündige Fußboten
\deffootnote{1.5em}{1em}{\makebox[1.5em][l]{\thefootnotemark}}

\typearea{14} % typearea berechnet einen sinnvollen Satzspiegel (das heißt die Seitenränder) siehe auch http://www.ctan.org/pkg/typearea. Diese Berechnung befindet sich am Schluss, damit die Einstellungen oben berücksichtigt werden
% für autoref von Gleichungen in itemize-Umgebungen
\makeatletter
\newcommand{\saved@equation}{}
\let\saved@equation\equation
\def\equation{\@hyper@itemfalse\saved@equation}
\makeatother 

\usepackage[framemethod=TikZ]{mdframed}
\usepackage{xcolor}

\definecolor{light-gray}{gray}{0.85}

\newmdenv[%
    backgroundcolor=light-gray,
    linecolor=light-gray,
    outerlinewidth=2pt,
    roundcorner=2mm,
    skipabove=\baselineskip,
    skipbelow=\baselineskip,
]{shaded}