% -*- coding: utf-8 -*-
% !TEX encoding = UTF-8 Unicode
% !TEX root =  main.tex

\chapter*{Einleitung}
\label{cha:einleitung}

Die vorliegende Projektarbeit wurde während des Wintersemesters 2014/2015 von Benjamin Ternes und Jan Feldkamp verfasst.
Inhaltlich setzt sich die Arbeit mit der Entwicklung von Simulationsblöcken für eine Regelung von permanentmagneterregten Synchronmaschinen auseinander.
Die Motivation dazu entstand im Rahmen der Vorlesung \glqq Aktorik und Leistungselektronik\grqq{} bei Herrn Prof. Dr. Arno Bergmann, im Masterstudiengang Elektrotechnik an der Hochschule Bochum.
Am Institut für Systemtechnik der Hochschule besteht der Bedarf eines vollständigen Simulationsmodells für eine feldorientierte Regelung einer PMSM.
Da das Institut eng mit der Elektromobilität, dem Schwerpunkt des Fachbereiches Elektrotechnik, verknüpft ist, spielt der Umgang mit PMSM eine wichtige Rolle.
Darüberhinaus sind hochdynamische Regelungen von Synchronmaschinen in nahezu jedem industriellen Bereich anzutreffen, was die Bearbeitung des Themenbereiches zusätzlich attraktiv macht.
Weiterhin stellt nicht nur die Antriebstechnik eine aktuelle Thematik in der Elektrotechnik dar, auch die modellbasierte Simulation ist ein etablierter und fester Bestandteil jedes Entwicklungsprozesses.
In der vorliegenden Arbeit werden beide Themenschwerpunkte miteinander verknüpft.

Ziel der Arbeit ist es zunächst, eine theoretische Grundlage für die Modellbildung einer anisotropen PMSM und für die Vektorregelung zu schaffen.
Dies geschieht mit einer Literaturrecherche.
Anschließend wird mit Hilfe dieser Grundlagen in \product{Matlab}/\product{Simulink} Simulationsblöcke für das Modell einer PMSM sowie für die feldorientierte Regelung erstellt.
Diese soll dem Institut als Basis für den Aufbau eines gesamten Regelungssystemes zur Verfügung gestellt werden.

Das Kapitel 1 beinhaltet die theoretischen Grundlagen zur Synchronmaschine, wobei hier von den Maxwellschen Gleichungen ausgegangen wird.
Im zweiten Kapitel sind die Grundlagen zur Vektorregelung enthalten, welche für die Koordinatentransformation benötigt werden.
Das dritte Kapitel beginnt mit einer Einführung in \product{Matlab}/\product{Simulink} und beinhaltet anschließend die Beschreibung der erstellten Simulationsblöcke.


 

\cleardoublepage
%%% Local Variables: 
%%% mode: latex
%%% TeX-master: "main"
%%% TeX-open-quote: "\\enquote{"
%%% TeX-close-quote: "}"
%%% LaTeX-csquotes-open-quote: "\\enquote{"
%%% LaTeX-csquotes-close-quote: "}"
%%% End: 